\section{Abstract}
Free Floating Car Sharing (FFCS) services are a flexible alternative to car ownership. These transportation services show highly dynamic usage both over different hours of the day, and across different city areas. 
In this work, we study the problem of predicting FFCS demand patterns -- a problem of great importance to an adequate provisioning of the service. We tackle both the prediction of the demand i) over time and ii) over space. 
We rely on months of real FFCS rides in Vancouver, which constitute our ground truth. We enrich this data with detailed socio-demographic information obtained from large open-data repositories to predict usage patterns. 
Our aim is to offer a thorough comparison of several machine learning algorithms in terms of accuracy and easiness of training, and to assess the effectiveness of current state-of-art approaches to address the prediction problem.
Our results show that it is possible to predict the future usage with relative errors down to 10\%, and the spatial prediction can be estimated with relative errors of about 40\%.
Our study also uncovered the socio-demographic features that most strongly correlate with FFCS usage, providing interesting insights for providers interested into opening service in new regions.

This work is mainly extracted from my paper \textit{On Car-Sharing Usage Prediction with Open Socio-Demographic Data}, published on the journal Electronics on January 2020 (\cite{cocca2020predictions}). The entire work was carried in collaboration with the Universidade Federal do Minas Gerais, Belo Horizonte, Brazil. My contribution are mainly related data collection, data augmenting and spatial analyses in sections \ref{sec:8_3_data_collection}, \ref{sec:8_3_datasetoverview} and \ref{sec:8_5_spatial analyses}.




