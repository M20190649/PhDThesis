\section{Key Performance Indicators and Simulation Scenario}
\label{sec:5_5_kpi_scenario}
In this section, I describe which are the simulation outputs and the scenario with which I performed the analyses. In particular I focused the attention on minimum requirements to system sustainability and measuring of users discomfort. 

\subsection{Performance metrics and parameters}

The simulator measures metrics that are  key to assess in the quality of experience for the customers:
\begin{itemize}
	\item \emph{Infeasible trip}: measures if a trip $i$ performed by a car $a$ ends with a completely discharged battery, i.e., when $c(a,t_{e}(i))= 0$;
	\item \emph{Charge event}: indicates a trip $i$ that ends with putting in charge the car, implying the burden to drive to the pole position, and plug the car;
	\item \emph{Reroute event}: a trip $i$ where the customer is rerouted to a zone different from the  desired destination because forced to charge the car $a$, i.e., $P(a,t_{e}(i))\neq D(i)$;
	\item \emph{Walk distance}: distance between the desired final location $d(i)$ and the actual final position $p(a,t_{end}(i))$.
\end{itemize}

The number of infeasible trips are critical, and the system shall be engineered so that they never happen. Other performance metrics shall be minimized. 
In addition to the above metrics, the simulator collects statistics about car battery charge level $c(a,t)$, and fraction of time a battery stays under charge.

\subsection{Simulation scenario}

I used this simulator to study the impact on the number of zones that are equipped with charging stations $N$, and the number of poles $k$ of each charging station.

I consider in each city a fleet that has a number of cars equal to the one observed in the trace. Electric cars have the same nominal characteristics as the Smart ForTwo Electric Drive, i.e., $17.6\,kWh$ battery, for $135\,km$ of range, with a discharge curve $Energy()$ that is proportional to the travelled distance ($12.9\,kWh/100\,km$). \footnote{\url{https://www.smart.com/uk/en/index/smart-electric-drive.html}} 
Charging stations have $k=4$ low power ($2\,kW$) poles each. These are cheap to install and a good compromise between costs, power requested, and occupied road section. We model a simple linear charge profile (complete charge in 8 hours and 50 minutes in our case). At last, the initial car position, only affecting the simulation transient, is chosen randomly.

The simulator, written in Python, takes less then 5 seconds to complete a single simulation for a given city and parameter set. 
Due to the large number of simulations, we run them in parallel. Each simulation produces 100\,MB of detailed logs, that we process on a Big Data cluster of 30 nodes using PySpark.%\footnote{\url{http://spark.apache.org/docs/latest/api/python/\#}}.