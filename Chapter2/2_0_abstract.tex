\section{Abstract}
Free Floating Car Sharing (FFCS) is a popular business model based on the shared-economy paradigm. The users can pick and drop the cars within a given operative area paying only for the time spent driving. This system implies front-end-crew-free reservation and releasing procedure, possible through a FFCS provider's web application. It exposes the coordinates of each available car and making them bookable by only one tap.

In this chapter, I describe how continuously monitoring two FFCS providers and how I created a dataset of FFCS trips. To do that,  I developed the \textit{Urban Mobility Analysis Platform} (\tool) able to fetches data from car sharing platforms in real time. Secondly, \tool processes the data to extract advanced information about driving patterns and user's habits. To extract information, \tool augments the data available from the car sharing platforms with mapping and direction information fetched from other web platforms. This information is stored in a data lake where historical series are built, and later analyzed using analytics modules easy to design and customize. 

In total \tool collected \mc{trips and bookinfs per c2g and engjoy for each city}

This chapter refers mostly my paper " \textit{UMAP: Urban mobility analysis platform to harvest car sharing data} \cite{ciociolaumap}, presented in at 2017 IEEE SmartWord conference


