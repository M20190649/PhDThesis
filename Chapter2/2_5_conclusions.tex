\section{Conclusions}
\label{sec:conclu}

In this study we presented \tool, a platform to collect and store data, and able to extract higher level information.
By means of two crawlers, we created a 52 days long dataset by collecting data for car2go and Enjoy, two different FFCS operators in the city of Turin.
By analyzing the data, we highlighted different aspects related to the system utilization, how users move in the city in different periods of the day, and what are the users' driving habits.
This analysis demonstrated how our platform can perform a wide range of analysis in a very simple way. By analyzing the system utilization, we demonstrated that FFCS cars are frequently used for short trips which last less then 30 minutes and 5 kms. We also demonstrated that, despite Enjoy has a smaller fleet, its system utilization is frequently higher than car2go one due to the more appreciated car model it offers. Exploiting the spatial analysis, we highlighted how users tend to move during different time periods. Finally, the users' driving habits showed us that current charging policy may encourage users to drive fast. 

The topic is worth further investigation. Thus we invite researchers that are interested to access our dataset and to use \tool that we make them available to the community.