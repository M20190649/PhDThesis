\section{Simulation scenarios }
\label{sec:7_4_modelling}

The goal is to study different design choices for electric car sharing systems, based on collected data. For this, we developed a flexible event-based simulator that allows us to compare different algorithms and tune  parameters while collecting metrics of interests.

\subsection{Simulation parameters}

Through the simulator widely described in chapter \ref{chap:5_simulator}, I simulate a fixed fleet of electric cars. Each car is characterised by its parking location, and the current status of battery charge. 
The simulator takes as input the pre-recorded data-set of rentals, i.e., the trace, characterised by the start and end time, and initial and final geographic coordinates.

Depending from the return policy, the customer may connect the car to a charging pole. , namely \textit{Free Floating}, \textit{Needed} and \textit{Hybrid}. The \textit{Free Floating} policy never obliges the customer to bring the car far from the desired ending location, even in case battery charge is close to exhaustion. \textit{Needed} mandates to connect cars to a charge station only if battery runs low, thus trying to protect from battery exhaustion. \textit{Hybrid} mixes the two policies letting customers opportunistically recharge the battery whenever they park close to a charging station. More information in section \ref{sec:5_4_return_policy}.

\reviewed{Notice that policies similar to \emph{Needed} have been introduced in~\cite{2_FlathIlgWeinhardt_2012}, where the system make the users charge the car considering the battery state of charge, the instantaneous electricity cost,  and the user's range anxiety.}



\subsection{Key Performances Indicators}
\label{sec:7_4_metrics}

I measure the following metrics, defined in section \ref{sec:5_5_simulation_scenario} that I identify having influence in the customers' quality of experience:
\begin{itemize}
	\item \textit{InfeasibleTrips\%}: percentage of infeasible trips due to completely discharged battery observed during the whole simulation;  
	\item \textit{Charges\%}: percentage of trips where the customer connects the car to a charging pole, implying the burden to plug the car;
	\item \textit{Reroutings\%}: percentage of trips where the customers are rerouted to a zone different from their original destination because they are forced to charge the car;
	\item \textit{WalkedDistance}: walked distance from the desired destination.
\end{itemize}

Infeasible trips are critical, and the system shall be engineered so that they never happen. Other performance metrics shall be minimised. 
In addition to the above metrics, the simulator collects statistics about car battery charge level, and fraction of time a battery stays under charge. 

The key design parameters that I focus on are (i) number of zones $Z$ which are equipped with a charging station; (ii) the locations of charging stations within the city; (iii) adopted return policies.

I consider the following scenario: the fleet has a constant number of cars equal to 377 (the same as observed in the trace).  Electric cars have the same nominal characteristics as the Smart ForTwo Electric Drive, i.e., $17.6\,kWh$ battery, for $135\,km$ of range, with a discharge curve that is proportional to the travelled distance ($12.9\,kWh/100\,km$).\footnote{\url{https://www.smart.com/uk/en/index/smart-electric-drive.html}} 
Charging stations have 4 low power ($2\,kW$) poles each. These are cheap to install and a good compromise between costs, power requested, and occupied road section. I model a simple linear charge profile (complete charge in 8 hours and 50 minutes in the case of study).


