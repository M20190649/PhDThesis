\section{Related Work}
\label{sec:10_2_related_work}

While first FFCS are operative since 2008 in Europe~\cite{Kortum2016}, the research on this topic has only recently flourished, especially for EVs.
%****************************************
%Analytical models
%***************************************
%\subsection{Modelling car sharing}
A common problem in transportation is to define models to optimize the fleet management and system design in general. For example, the authors of~\cite{11_nair2011fleet} proposed a Mixed Integer Problem (MIP) to maintain and organize fleet distribution in short term considering a stochastic demand. In the recent work~\cite{5_he2017service} authors showed how to analytically model customers' probability to use car sharing. %  and imbalanced travel patterns can tackle the fleet management and relocation challenge in the short-term temporal horizon.  
Other studies include more complex phenomena in their works, like \cite{16_xu2019fleet} where the authors consider a non-linear charging function and detailed power lines constraints to optimally design \emph{one-way} car sharing system (using a MIP). %Then, they tested the model in Singapore with promising results.  

%****************************************
%Behaviour simulation
%***************************************
%\subsection{Simulating car-sharing from real data}
Another strategy to study FFCS is to \emph{simulate} how users interact with it. For example, the authors of \cite{13_ciari2008concepts} proposed (but did not implement) an agent simulator approach to measure how FFCS can be scaled on the entire Swiss traffic. 
The authors of~\cite{12_martin2011greenhouse} present a study of two-ways car sharing growth, with the help of an event-based simulator that measures if and how charging stations produce profits. 
Similarly, \cite{24_ciari2016modeling} proposes an open-source multi-agent simulator able to replicate travelling people's habits. In particular, it focuses on the realistic replication of users' behavioral model relying on multinomial distribution of modal choice.

Recently, the availability and abundance of data helped shaping FFCS users' behaviour.  Considering this, some works like~\cite{9_habibi2017comparison} and~\cite{10_ciociola2017umap} scraped data from real ICE FFCS, characterized their services and proposed model generalizations. %In particular both showed how median travel distance and time strongly depend on city conformation. 
Big Data approaches helped researchers to improve the simulation fidelity. In particular, authors of~\cite{0_tachet2017scaling} used data to predict the shareability of an urban ride, finding that this a property city-invariant.  %Surprisingly it depends only trips-related characteristics like trips per hour or average traffic speed and not the city shape. The formula is tested in several cities like San Francisco, Vienna, New York and Singapore, producing accurate results. 
On the same optic, the authors of~\cite{22_li2016design} use data from several American cities to optimize the position of the charging stations of a one-way car sharing, finding that  the optimal  results place the stations in high-demand areas.  
This results in this chapters confirms the results in \cite{7_cocca2019free} and \cite{8_cocca2019free} where I used a simulation based approach to measure the impact of different design options of an EVs FFCS system. %The studies confirmed that best results come from placing charging station high-demand areas. 
Here, the use of big data to derive realistic demand models to feed accurate simulations. I move one step forward - showing that this is beneficial also as a proxy of relocation, i.e., cars get naturally relocated to high demand zones.

%****************************************
%Economic sustainability
%***************************************
%\subsection{Economic sustainability}
The economic sustainability is another key aspect of car sharing - especially with EVs. Authors of~\cite{20_hua2019joint} studied the economic sustainability of one way electric car sharing systems finding out that charging station should have an amortization period of at least 5 years to produce profits.
The authors of~\cite{19_lemme2019optimization} compared how FFCS with EVs and ICE can produce profits, observing the best compromises with ICE fuelled with cheap and cleaner fuel like ethanol. 
Another study~\cite{21_vasconcelos2017financial} concerning the city of Lisbon found out that switching to EVs would cost more than ICE and would lead to a negative profit of about one million per year. This is largely due to the higher cost of electric cars and of the charging infrastructure.
This chapter explores which are the most efficient and economically sustainable combinations of fleet size, charging infrastructure design, also in light of demand growth.

%****************************************
%Difference of our work
%***************************************
To the best of I can tell, this work is among the first to study the scalability of a FFCS system with electric vehicles, exploring key parameters like number of poles, fleet size and increase in demand can affects economic and performance of the system.


