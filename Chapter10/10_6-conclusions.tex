\section{Conclusions}
\label{sec:10_6_conclusions}


In this chapter I presented a simulation study of free floating car sharing systems. Armed with a realistic demand, I studied the performance implications of moving from ICE FFCS to a EV based solution.

This study offers several take-away messages: first, the charging infrastructure must be able to provide enough energy to cope with the mobility demand. Interestingly, it results to be quite limited, with just 20 poles able to sustain a system of 400 vehicles.
Second, distributing the charging poles in zones where cars get frequently parked and rented is instrumental to maximize the demand the system can satisfy, while also limiting the time workers have to spend to bring cars for charging.
Third, the system exhibits useful economy of scale, so that the fleet size shall increase sublinearly with respect to the mobility demand intensity. 

At last, when projected into economic figures, the fleet setup and management represent the main cost factors. Choosing the right number of vehicles results more fundamental than optimizing the charging infrastructure costs. For instance, for the current demand intensity in Turin, the switch to EVs must be carefully designed to be profitable. Interestingly, when the demand grows, the margins are much higher, allowing some nice economy of scale opportunities.

As future direction, I are studying different cities as new use cases, looking at opportunities of involving users in the charging process in order to decrease management costs.